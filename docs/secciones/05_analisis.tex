\chapter{Análisis del problema}
 
En esta sección se mostrarán las pruebas que se han realizado para dar por solucionado los distintos problemas que plantea resolver 
el proyecto, junto a un análisis de los resultados obtenidos.

\section{Creación de equipos igualados}

El primer problema que plantea resolver el proyecto es la creación de 2 equipos igualados en función del nivel de los jugadores disponibles.
Para comprender los resultados de los análisis, se deben conocer los siguientes conceptos:

\begin{itemize}
    \item El nivel de cada jugador es un entero que puede oscilar entre el 1 y el 100.
    \item El nivel de un equipo viene dado por la media de nivel de los jugadores que lo componen.
    \item Dos equipos se consideran igualados si la diferencia de nivel entre ambos equipos no es superior a 10 niveles.
\end{itemize}

\newpage

El primer análisis efectuado pretende comprobar si el algoritmo establecido para la creación de equipos mejora la aleatoriedad.\\

Se ha creado aleatoriamente una lista de 10 elementos comprendidos entre el 1 y el 100 que hace referencia al nivel de los 10 jugadores
que se simulan en esta prueba y se han realizado 5 combinaciones aleatorias. La lista ha sido la siguiente:  73, 70, 42, 66, 28, 16, 29, 50, 10, 18.\\


\begin{tabular}{| l | l | c |}
    \hline
    \textbf{Equipo1} & \textbf{Equipo2} & \textbf{Diferencia de nivel}\\
    \hline
    18, 29, 70, 73, 16 & 50, 10, 42, 66, 28 & 2\\
    \hline
    10, 50, 18, 16, 29 & 70, 66, 28, 73, 42 & 31\\
    \hline
    16, 73, 70, 28, 66 & 10, 42, 18, 50, 29 & 21\\
    \hline
    73, 29, 10, 16, 28 & 66, 18, 70, 50, 42 & 18\\
    \hline
    66, 18, 16, 70, 42 & 73, 29, 10, 50, 28 & 4\\
    \hline
\end{tabular}\\


Podemos ver que la aleatoriedad nos da un resultado de diferencia de nivel medio de 15,2.\\

Sin embargo, si con estos mismos datos de entrada aplicamos el algoritmo diseñado para nuestro proyecto, obtenemos los siguientes resultados:\\

\begin{tabular}{| l | l | c |}
    \hline
    \textbf{Equipo1} & \textbf{Equipo2} & \textbf{Diferencia de nivel}\\
    \hline
    73, 50, 42, 18, 16 & 70, 66, 29, 28, 10 & 1\\
    \hline
\end{tabular}\\

Podemos observar que el algoritmo implementado supera con creces la aleatoriedad.
