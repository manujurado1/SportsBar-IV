\chapter{Descripción del problema}

El problema que he encontrado, observando mi día a día y que pasa en mi grupo de amigos, y soy consciente que pasa en la mayoría de los grupos de personas que se juntan para jugar partidos entre amigos, 
es que siempre se generan conflictos a la hora de crear los equipos para jugar el partido. 
En estos casos se busca repartir a todos los jugadores disponibles en 2 equipos que sean lo más igualados posibles, ya que se suele dar la situación de que la diferencia de nivel es bastante grande entre unos y otros,
ya que quizás algunos han practicado este deporte muchos años y otros solo lo practican para distraerse de vez en cuando.
Esta igualdad se busca, sobre todo, para que todas las personas disfruten en ese rato con los amigos o compañeros, ya que nadie disfruta ganando fácil al rival y mucho menos siendo aplastado. Por eso, cuando en estos encuentros se consigue
un partido igualado, la experiencia de juego de todos los participantes es mayor. 