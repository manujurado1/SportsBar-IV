\chapter{Estado del arte}

El software libre y sus licencias \cite{gplv3} ha permitido llevar a cabo una expansión del
aprendizaje de la informática sin precedentes.\\

En esta sección se pretende hacer un estudio sobre las diferentes alternativas
que encontramos en el panorama para resolver cada uno de los problemas que plantea
resolver este proyecto.

Gracias a esto, podremos ver los puntos fuertes y las debilidades del panorama relacionado
con las funciones de nuestro proyecto, y de esta manera poder darle un mejor enfoque.

\section{Análisis del mercado}

\subsection{Creación de equipos}

En el mercado existen una gran cantidad de aplicaciones para crear equipos, tanto en forma de app como en la web,
ya que estas se dedican simplemente a dividir aleatoriamente una lista en 2 listas de igual tamaño o, en otros casos,
en crear el número de listas necesarias una vez indicado el tamaño que deben tener estas listas.
Muchas de estar poseen una interfaz basada en el deporte, lo que se convierte en generar equipos en vez de listas como tal.\\

Este tipo de aplicaciones resolverían una pequeña parte de lo que nuestro proyecto plantea resolver, ya que solo crean equipos
y estas no están enfocadas en que los equipos estén igualados ni en gestionar los niveles de cada jugador individualmente.

\newpage

\subsection{Gestión de niveles por jugador}

Respecto a la parte del problema que planteamos resolver la cual no abarcan las opciones mencionadas anteriormente, entre las que
se encuentra la gestión de niveles, no se ha encontrado nada parecido enfocado a la creación de equipos.

Sin embargo, hay un tipo de juegos que están en auge en la actualidad que tienen en cuenta de cierta manera el nivel de los
jugadores en función de su desempeño.

Este tipo de juegos tratan de hacer equipos con los jugadores de la primera división española de fútbol y conseguir puntos en función
de su rendimiento real en los partidos. De esta manera obtienen una puntuación de cada jugador en función de su desempeño.\\

Enfocándonos a nuestro objetivo, se pretende adecuar el nivel de cada jugador en función a su desempeño en cada partido, pero en nuestro caso,
el único factor a tener en cuenta será el resultado, ya que no es viable manejar la cantidad de factores que manejan este tipo de aplicaciones,
que al enfocarse en partidos oficiales, consiguen las estadísticas fácilmente.

\section{Mi propuesta ante el estado del arte}

Tras analizar como se están gestionando actualmente en el mercado la resolución de los problemas que plantea resolver nuestro proyecto, he obtenido
una idea más concisa de cómo enfocar la solución al problema planteado.

Esta consta de los siguientes puntos:

\begin{itemize}
    \item Creación de equipos dada una lista de jugadores, pero en este caso, teniendo en cuenta el nivel de los mismos para que los equipos sean igualados.De esta manera, conseguiremos diferenciarnos
    del resto de aplicaciones añadiendo un extra de valor.
    \item Tener en cuenta el resultado de los partidos para modificar o no el nivel de cada jugador en consecuencia de este.
\end{itemize}
