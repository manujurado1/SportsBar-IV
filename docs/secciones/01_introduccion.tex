\chapter{Introducción}

Este proyecto es software libre, y está liberado con la licencia \cite{gplv3}.

\section{Motivación}

Mi principal motivación para realizar este proyecto fue buscar problemas o impedimentos que me surgieran en mi día a día.
En mi caso personal, debido a que estudio fuera de casa, siempre que vuelvo a mi pueblo, busco quedar con los amigos que se encuentren también esos días en el pueblo, 
normalmente para jugar un partido de fútbol entre nosotros, ya que es uno de nuestros pasatiempos preferidos.\\

Y aquí me di cuenta que 2 problemas que surgían en estos encuentros:
\begin{enumerate}
    \item Gastábamos gran parte del tiempo con el fin de repartirnos en equipos igualados, puesto que la diferencia de nivel es bastante grande entre unos y otros.
    \item Si no conseguíamos que el partido estuviera igualado, nos cansábamos antes de jugar, ya que la igualdad y la competitividad nos mantenía más enchufados al partido.
\end{enumerate}

Entonces, a partir de este par de problemas, pensé en alguna forma de solucionarlos, ahorrando, por una parte, el tiempo que gastábamos en crear los equipos y, por otra parte,
aumentando lo máximo posible las posibilidades de que el encuentro estuviera igualado.

\newpage

\section{Objetivos}

Teniendo en cuenta los problemas encontrados, se enfocará la aplicación en resolverlos de la siguiente manera:
\begin{itemize}
    \item Dando la posibilidad al usuario de crear un grupo de juego, con tantos jugadores como desee, incluyendo en estos un nivel inicial aproximado del jugador.
    \item Permitiendo elegir antes de cada partido, qué jugadores del grupo están disponibles para jugar ese partido.
    \item Después de cada partido, regulando los niveles de los jugadores en función del resultado para que el nivel de cada jugador en la aplicación esté cada vez más cerca de su nivel real.
\end{itemize}