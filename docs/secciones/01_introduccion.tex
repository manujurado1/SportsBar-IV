\chapter{Introducción}

Este proyecto es software libre, y está liberado con la licencia \cite{gplv3}.

\section{Motivación}

Mi principal motivación para realizar este proyecto fue buscar problemas o impedimentos que me surgieran en mi día a día.
En mi caso personal, debido a que estudio fuera de casa, siempre que vuelvo a mi pueblo, busco quedar con los amigos que se encuentren también esos días en el pueblo, 
normalmente para jugar un partido de fútbol entre nosotros, ya que es uno de nuestros pasatiempos preferidos.\\

\section{Problema encontrado}

El problema que he encontrado, observando mi día a día y que pasa en mi grupo de amigos, y soy consciente que pasa en la mayoría de los grupos de personas que se juntan para jugar partidos entre amigos, 
es que siempre se generan conflictos y se pierde una gran cantidad de tiempo a la hora de crear los equipos para jugar el partido. 
En estos casos se busca repartir a todos los jugadores disponibles en 2 equipos que sean lo más igualados posibles, ya que se suele dar la situación de que la diferencia de nivel es bastante grande entre unos y otros,
ya que quizás algunos han practicado este deporte muchos años y otros solo lo practican para distraerse de vez en cuando.
Esta igualdad se busca, sobre todo, para que todas las personas disfruten en ese rato con los amigos o compañeros, ya que nadie disfruta ganando fácil al rival y mucho menos siendo aplastado. Por eso, cuando en estos encuentros se consigue
un partido igualado, la experiencia de juego de todos los participantes es mayor. 

\section{Personas a las que les afecta este problema}

Este problema ocurre en la mayoría de aficionados del deporte que practican algún deporte de equipo como hobbie.

Aquí expongo el caso de 1 persona y cómo afecta este problema en su día a día:

\begin{itemize}
    \item \textbf{Javier Ruiz:} Estudiante de farmacia que ha tenido que mudarse a una ciudad para estudiar, el cuál cuando vuelve a su pueblo,
     se reúne con sus amigos a jugar un partido de fútbol como cuando compartían clase. Debido a que sus amigos suelen estar muy ocupados,
     no siempre se reúnen los mismos amigos a jugar. Normalmente pierden bastante tiempo en formar los equipos buscando que el partido esté igualado,
     ya que aún mantienen el pique de demostrar quien es el mejor jugador de fútbol de la pandilla. 
    
\end{itemize}
