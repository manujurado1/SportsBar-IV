\chapter{Metodología utilizada}

Para este proyecto se ha usado metodologías ágiles basadas en SCRUM. No ha sido posible seguir SCRUM de una manera más precisa, ya que este proceso está enfocado a trabajar en
equipo, en entornos complejos y orientado a la existencia de clientes. En este caso, al ser un único desarrollador y un entorno tan complejo, el proceso basado en SCRUM se ha
focalizado más en buscar realizar entregas parciales y regulares hacia el producto final. Esto se ha llevado a cabo convirtiendo las historias de usuario en una serie de hitos
que se deben ir alcanzando para tener la historia de usuario completa.\\

\section{Planificación}

Para la planificación de este proyecto, se ha partido de historias de usuario las cuales se desean complacer. A partir de estas, se han creado una serie de hitos que conformaban productos mínimamente viables,
 que nos llevan hasta el producto final.

\subsection{Historias de usuario}

Para conseguir las historias de usuario sobre las cuales se crearán los hitos, hay que personificar tipos de usuarios que necesitan nuestra aplicación, ya que se encuentran en su día a día
con problemas los cuales se resuelven con la misma.

\newpage

\begin{itemize}
    \item \textbf{Javier Gómez}: 31 años, profesor de primaria. Es un gran aficionado al fútbol sala, y práctica este deporte todas las semanas junto a sus amigos de la infancia. No buscan ningún tipo de competitividad ni buscan ganar los partidos, usan este momento para despejar la mente y reencontrarse.
    \item \textbf{Carmen Ruiz}: 18 años, estudiante universitaria. Lleva jugando a fútbol desde que era pequeña y ahora que ha entrado a la universidad, se ha unido a una peña de personas que les gusta el fútbol, pero de nivel muy variado, para seguir practicando y mejorando el deporte que tanto le gusta.
    \item \textbf{Miguel Zamora}: 23 años, camarero. Su grupo de amigos está formado por hombres que de pequeños compartían vestuario en un equipo de alto nivel, pero la vida los terminó llevando por otros caminos. Juegan a fútbol con un alto grado de competitividad, siempre pensando que son la estrella del equipo, y normalmente incluso apuestan económicamente en favor a ese dato.
\end{itemize}

Y una vez tenemos las personas, ya se pueden crear las historias de usuario:

\begin{itemize}
    \item Javier quiere crear un grupo de juego con sus amigos de la infancia y obtener 2 equipos totalmente al azar una vez haya seleccionado quienes de sus amigos pueden asistir a esa sesión. No importa el nivel de cada uno, ya que el fútbol es sólo la excusa para juntarse.
    \item Carmen y sus compañeras de peña, las cuales juegan a fútbol por diversión y dónde se encuentra gente de todo tipo de nivel, quieren crear equipos igualados en función del nivel estimado de cada persona que participa para que el partido esté disputado y todo el mundo pueda aprender y disfrutar.
    \item Miguel y sus amigos futboleros, con los que juega partidos a menudo, están en continua competición de demostrar quién es el mejor jugador de fútbol de la pandilla, después de ninguno de ellos haber conseguido triunfar en este deporte. Quieren buscar alguna manera de tener un historial de todos sus enfrentamientos, tener partidos igualados y encontrar alguna forma de poder medir lo buenos que son respecto a sus amigos para demostrar quien manda.
\end{itemize}

\newpage
\section{Temporización}

La temporización del proyecto se ha realizado por medio de \textit{sprints}.\\

En cada uno de estos \textit{sprints} se han ido fijando una serie de problemas a resolver, todos ellos enfocados a satisfacer las historias de usuario, avanzando en cada momento
alguno de los hitos mencionados anteriormente.


\section{Seguimiento del desarrollo}

Con el fin de obtener una mayor trazabilidad de cómo se ha ido desarrollando el proyecto, se ha usado Git como sistema de control de versiones, el cual nos permite ver versiones anteriores
del proyecto, revertir cambios y progresar en el proyecto por distintas ramas, lo que aumenta el grado de adaptabilidad.\\

Para alojar este proyecto se ha usado GitHub. Esta plataforma permite observar en todo momento cuál es el estado actual del proyecto y además ofrece herramientas para el control de calidad
a través de las \textit{GitHub Actions}, de las que hablaremos a continuación.


\section {Control de calidad}

Para dotar al proyecto de calidad se han implementado junto al desarrollo del mismo una serie de \textit{tests} siguiendo la práctica de desarrollo dirigido por pruebas o TDD.\\

Siguiendo esta práctica, se ha priorizado la creación de las pruebas, escribiendo el código fuente tras estas. De esta manera, se da el código por válido cuando este supera sus pruebas asociadas.
De esta forma se crea una batería de pruebas, la cual empieza con una base de pruebas unitarias y acaba con pruebas de integración y \textit{End-to-end}.\\

Estos controles de calidad han sido muy provechosos al fusionarlos con la integración continua, práctica de desarrollo software mediante la cuál se ejecutan pruebas automáticas sobre el código cuando este es modificado.
La integración continua se ha conseguido usando \textit{GitHub Actions} y \textit{Checks API}, ambas herramientas provenientes de GitHub.